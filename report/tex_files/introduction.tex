
\section{Introducción}
		La palabra pandemia ha ocupado un lugar muy importante en nuestras vidas este último año. El virus COVID-19 se ha convertido en la mayor preocupación mundial en la actualidad, no solo tiene consecuencias en la salud de millones de personas alrededor del mundo, sino que también esta situación y esta pandemia mundial ha producido muchos otros efectos negativos. Ningún gobierno, ni organización, ni persona está preparada para sobrellevar una pandemia mundial. Por lo que no solo ha afectado a la salud de miles de personas sino que también ha afectado a la economía, a la sociedad, a la política y las grandes potencias mundiales han actuado a ciegas ya la información sobre el mismo era escasa.\\
		
		\noindent La primera secuencia que se obtuvo del genoma del agente infeccioso pudo ser encontrada en enero del año siguiente. Esto fue crucial para identificar al virus como un coronavirus, encontrando similitudes al coronavirus responsable del Síndrome Agudo Respiratorio Grave (SARS), la enfermedad respiratoria nacida en Asia en 2003, la cual también se convirtió en pandemia. Es por ello que es tan importante el conocimiento de los virus de SARS y MERS. El tamaño de los viriones de SARS-CoV-2 es de aproximadamente 50 a 200 nm de diámetro y su genoma está formado por ARN monocatenario de sentido positivo. La secuencia del betacoronavirus de Wuhan, de aproximadamente treinta mil nucleotidos de longitud, se relacionó por parecido con los betacoronavirus que afectaban a los murciélagos, pero son genéticamente diferenciables de otros coronavirus como el SARS-CoV y el MERS-CoV.  Está compuesto de cuatro genes para las proteínas estructurales que caracterizan a los coronavirus, los cuales se identifican mediante las letras S (homotrímero de glicoproteína que forman las puntas de la superficie), E (proteína de bajo tamaño de la envoltura), M (proteína de la matriz que une la envoltura con el núcleo) y N (fosfoproteína de la nucleocápside), además de los marcos de lectura abiertos que codifican proteínas no estructurales en las que encontramos, las enzimas causantes de su ciclo reproductivo intrahospedero. \\
		
		\noindent Toda esta información sobre el virus era desconocida en diciembre de 2019 cuando apareció en Wuhan , provincia de Hubei (China). Un brote epidémico de lo que se llamaba neumonía por causa desconocida que llegó a afectar a más de 60 personas durante ese mes. Esto es debido a que el coronavirus puede infectar de manera selectiva las mucosas pulmonares o grastrointestinales. La forma de acceder a una célula epitelial es mediante un receptor presente en las superficie del organismo que recibe el nombre de ACE2. Dichos receptores son más comunes ser encontrados en los pulmones, por ello esta enfermedad está considerada de tipo respiratorio. El sistema inmunológico humano contraataca con una respuesta dura, liberando interferones, cuya función es dificultar la replicación del virus dentro de las células epiteliales.\\
		
		
		\noindent En este trabajo, estudiaremos la respuesta desarrollada en las células del epitelio del pulmón a la infección por SARS-Cov2 mediante el análisis de los perfiles de expresión génica publicados en el dataset GEO GSE147507. Estos perfiles de expresión se analizarán mediante el modelado de redes de coexpresión génica, con el paquete de R WGCNA.
		

	

