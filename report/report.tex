\documentclass{bmcart}

%%%%%%%%%%%%%%%%%%%%%%%%%%%%%%%%%%%%%%%%%%%%%%
%%                                          %%
%% CARGA DE PAQUETES DE LATEX               %%
%%                                          %%
%%%%%%%%%%%%%%%%%%%%%%%%%%%%%%%%%%%%%%%%%%%%%%

%%% Load packages
\usepackage{amsthm,amsmath}
\usepackage{graphicx}
%\RequirePackage[numbers]{natbib}
%\RequirePackage{hyperref}
\usepackage[utf8]{inputenc} %unicode support
%\usepackage[applemac]{inputenc} %applemac support if unicode package fails
%\usepackage[latin1]{inputenc} %UNIX support if unicode package fails


%%%%%%%%%%%%%%%%%%%%%%%%%%%%%%%%%%%%%%%%%%%%%%
%%                                          %%
%% COMIENZO DEL DOCUMENTO                   %%
%%                                          %%
%%%%%%%%%%%%%%%%%%%%%%%%%%%%%%%%%%%%%%%%%%%%%%

\begin{document}

	\begin{frontmatter}
	
		\begin{fmbox}
			\dochead{Research}
			
			%%%%%%%%%%%%%%%%%%%%%%%%%%%%%%%%%%%%%%%%%%%%%%
			%% INTRODUCIR TITULO PROYECTO               %%
			%%%%%%%%%%%%%%%%%%%%%%%%%%%%%%%%%%%%%%%%%%%%%%
			
			\title{Respuesta a la Infección por SARS-Cov2}
			
			%%%%%%%%%%%%%%%%%%%%%%%%%%%%%%%%%%%%%%%%%%%%%%
			%% AUTORES. METER UNA ENTRADA AUTHOR        %%
			%% POR PERSONA                              %%
			%%%%%%%%%%%%%%%%%%%%%%%%%%%%%%%%%%%%%%%%%%%%%%
			
			\author[
			  addressref={aff1},                   % ESTA LINEA SE COPIA IGUAL PARA CADA AUTOR
			  corref={aff1},                       % ESTA LINEA SOLO DEBE TENERLA EL COORDINADOR DEL GRUPO
			  email={jane.e.doe@cambridge.co.uk}   % VUESTRO CORREO ACTIVO
			]{\inits{S.C.M}\fnm{Susana} \snm{Cano Marín}} % inits: INICIALES DE AUTOR, fnm: NOMBRE DE AUTOR, snm: APELLIDOS DE AUTOR
			\author[
			  addressref={aff1},
			  email={john.RS.Smith@cambridge.co.uk}
			]{\inits{J.S.R.}\fnm{Juan} \snm{Sánchez Rodríguez}}
			
			\author[
			addressref={aff1},
			email={laurafg98@uma.es}
			]{\inits{J.S.R.}\fnm{Laura} \snm{Fernández García}}
			
			\author[
			addressref={aff1},
			email={laurafg98@uma.es}
			]{\inits{J.S.R.}\fnm{Mariana} \snm{Gonzalez}}
			
			\author[
			addressref={aff1},
			email={laurafg98@uma.es}
			]{\inits{J.S.R.}\fnm{Ana} \snm{}}
			
			%%%%%%%%%%%%%%%%%%%%%%%%%%%%%%%%%%%%%%%%%%%%%%
			%% AFILIACION. NO TOCAR                     %%
			%%%%%%%%%%%%%%%%%%%%%%%%%%%%%%%%%%%%%%%%%%%%%%
			
			\address[id=aff1]{%                           % unique id
			  \orgdiv{ETSI Informática},             % department, if any
			  \orgname{Universidad de Málaga},          % university, etc
			  \city{Málaga},                              % city
			  \cny{España}                                    % country
			}
		
		\end{fmbox}% comment this for two column layout
		
		\begin{abstractbox}
		
			\begin{abstract} % abstract
			
			%%%%%%%%%%%%%%%%%%%%%%%%%%%%%%%%%%%%%%%%%%%%%%%
			%% RESUMEN BREVE DE NO MAS DE 100 PALABRAS   %%
			%%%%%%%%%%%%%%%%%%%%%%%%%%%%%%%%%%%%%%%%%%%%%%%	
			
			\end{abstract}
			
			%%%%%%%%%%%%%%%%%%%%%%%%%%%%%%%%%%%%%%%%%%%%%%
			%% PALABRAS CLAVE DEL PROYECTO              %%
			%%%%%%%%%%%%%%%%%%%%%%%%%%%%%%%%%%%%%%%%%%%%%%
			
			\begin{keyword}
			\kwd{sample}
			\kwd{article}
			\kwd{author}
			\end{keyword}
		
		
		\end{abstractbox}
	
	\end{frontmatter}
	
	%%%%%%%%%%%%%%%%%%%%%%%%%%%%%%%%%%%%%%%%%%%%%%%%%%%%%%%%%%%%%%%%%%%%%%%%%%%%%%%%%%%%%%%%
	%% EJEMPLO DE LATEX %%                                                                %%
	%% BORRAR ANTES DE ENTREGAR!!!!!!!!!!!!!!!!!!!!!!!!!!!!!!!!!!!!!!!!!!!!!              %%
	%%%%%%%%%%%%%%%%%%%%%%%%%%%%%%%%%%%%%%%%%%%%%%%%%%%%%%%%%%%%%%%%%%%%%%%%%%%%%%%%%%%%%%%%
	
	\section*{Content}
		Text and results for this section, as per the individual journal's instructions for authors. Here, we reference the figure \ref{fig:cost_genome} and figure \ref{fig:cost_megabase} but also the table \ref{tab:ejemplo}.
	
	\section*{Section title}
		Text for this section\ldots

		In this section we examine the growth rate of the mean of $Z_0$, $Z_1$ and $Z_2$. In
		addition, we examine a common modeling assumption and note the
		importance of considering the tails of the extinction time $T_x$ in
		studies of escape dynamics.
		We will first consider the expected resistant population at $vT_x$ for
		some $v>0$, (and temporarily assume $\alpha=0$)
		%
		\[
		E \bigl[Z_1(vT_x) \bigr]=
		\int_0^{v\wedge
			1}Z_0(uT_x)
		\exp (\lambda_1)\,du .
		\]
		%
		If we assume that sensitive cells follow a deterministic decay
		$Z_0(t)=xe^{\lambda_0 t}$ and approximate their extinction time as
		$T_x\approx-\frac{1}{\lambda_0}\log x$, then we can heuristically
		estimate the expected value as
		%
		\begin{equation}\label{eqexpmuts}
			\begin{aligned}[b]
				&      E\bigl[Z_1(vT_x)\bigr]\\
				&\quad      = \frac{\mu}{r}\log x
				\int_0^{v\wedge1}x^{1-u}x^{({\lambda_1}/{r})(v-u)}\,du .
			\end{aligned}
		\end{equation}
		%
		%%%%%%%%%%%%%%%%%%%%%%%%%%%%%%%%%%%%%%%%%%%%%%%%%%%%%%%%%%%%%%%%%%%%%%
		%% USAR \cite{...} PARA INCLUIR REFERENCIAS BIBLIOGRAFICAS          %%
		%%  \cite{koon}  Para una sola                                      %%
		%%  \cite{oreg,khar,zvai,xjon,schn,pond} Para una lista             %%
		%%%%%%%%%%%%%%%%%%%%%%%%%%%%%%%%%%%%%%%%%%%%%%%%%%%%%%%%%%%%%%%%%%%%%%
		Thus we observe that this expected value is finite for all $v>0$ (also see \cite{koon,xjon,marg,schn,koha,issnic}).
		
		
		%%%%%%%%%%%%%%%%%%%%%%%%%%%%%%%%%%%%%%%%%%%%%%%%%%%%%%%%%%%%%%%%%%%%%%%%%%%%%%%%%%%%%%%%%%%
		%% FIGURAS                                                                               %%
		%% includegraphics: inserta la imagen                                                    %%
		%% caption: descripcion de la figura                                                     %%
		%% label: etiqueta para hacer referencia a la figura en el texto con la instrucción ref  %%
		%%%%%%%%%%%%%%%%%%%%%%%%%%%%%%%%%%%%%%%%%%%%%%%%%%%%%%%%%%%%%%%%%%%%%%%%%%%%%%%%%%%%%%%%%%%	
		
		\begin{figure}[h!]
			\includegraphics[width=0.9\textwidth]{figures/Sequencing_Cost_per_Genome_May2020.jpg}
			\caption{Sample figure title}
			\label{fig:cost_genome}
		\end{figure}
		
		\begin{figure}[h!]
			\includegraphics[width=0.9\textwidth]{figures/Sequencing_Cost_per_Megabase_May2020.jpg}
			\caption{Sample figure title}
			\label{fig:cost_megabase}
		\end{figure}
		
		%%%%%%%%%%%%%%%%%%%%%%%%%%%%%%%%%%%%%%%%%%%%%%%%%%%%%%%%%%%%%%%%%%%%%%%%%%%%%%%%%%%%%%%%%%
		%% TABLAS                                                                               %%
		%% caption: Descripción tabla                                                           %%
		%% \begin{tabular}{letras}: Indica numero de columnas.                                  %%
		%%    Una letra por columna, la letra indica la alineación de la columna:               %%
		%%    c center, l left, r right                                                         %%
		%% hline: Representa una linea como entre filas                                         %%
		%% \\: fin de fila                                                                      %%
		%% &: delimitador de celda                                                              %%
		%% label: etiqueta para hacer referencia a la tabla en el texto con la instrucción ref  %%
		%%%%%%%%%%%%%%%%%%%%%%%%%%%%%%%%%%%%%%%%%%%%%%%%%%%%%%%%%%%%%%%%%%%%%%%%%%%%%%%%%%%%%%%%%%
		
		\begin{table}[h!]
			\caption{Sample table title. This is where the description of the table should go}
			\begin{tabular}{cccc}
				\hline
				& B1  &B2   & B3\\ 
				\hline
				A1 & 0.1 & 0.2 & 0.3\\
				A2 & ... & ..  & .\\
				A3 & ..  & .   & .\\ 
				\hline
				\label{tab:ejemplo}
			\end{tabular}
		\end{table}
				
		\subsection*{Sub-heading for section}
			Text for this sub-heading\ldots
	
			\subsubsection*{Sub-sub heading for section}
				Text for this sub-sub-heading\ldots
					
					\paragraph*{Sub-sub-sub heading for section}
						Text for this sub-sub-sub-heading\ldots
	
	%%%%%%%%%%%%%%%%%%%%%%%%%%%%%%%%%
	%% FIN DE EJEMPLO !!!!!!!!!!!! %%
	%%%%%%%%%%%%%%%%%%%%%%%%%%%%%%%%%
	
	%%%%%%%%%%%%%%%%%%%%%%%%%%%%%%%%%
	%% COMIENZO DEL DOCUMENTO REAL %%
	%%%%%%%%%%%%%%%%%%%%%%%%%%%%%%%%%
	
	\documentclass{article}


\begin{document}
	
	\section{Introducción:}
		La palabra pandemia ha ocupado un lugar muy importante en nuestras vidas este último año. El virus COVID-19 se ha convertido en la mayor preocupación mundial en la actualidad, no solo tiene consecuencias en la salud de millones de personas alrededor del mundo sino que también esta situación y esta pandemia mundial ha producido muchos otros efectos negativos. Ningún gobierno, ni organización, ni persona está preparada para sobrellevar una pandemia mundial. Por lo que no solo ha afectado a la salud de miles de personas sino que también ha afectado a la economía, a la sociedad, a la política etc y las grandes potencias mundiales han actuado a ciegas ya la información sobre el mismo era escasa.\\
		
		La primera secuencia que se obtuvo del genoma del agente infeccioso pudo ser encontrada en enero del año siguiente. Esto fue crucial para identificar al virus como un coronavirus, encontrando similitudes al coronavirus responsable del Síndrome Agudo Respiratorio Grave (SARS), la enfermedad respiratoria nacida en Asia en 2003, la cual también se convirtió en pandemia. Es por ello que es tan importante el conocimiento de los virus de SARS y MERS. El tamaño de los viriones de SARS-CoV-2 es de aproximadamente 50 a 200 nm de diámetro y su genoma está formado por ARN monocatenario de sentido positivo. La secuencia del betacoronavirus de Wuhan, de aproximadamente treinta mil nucleotidos de longitud, se relacionó por parecido con los betacoronavirus que afectaban a los murciélagos, pero son genéticamente diferenciables de otros coronavirus como el SARS-CoV y el MERS-CoV.  Está compuesto de cuatro genes para las proteínas estructurales que caracterizan a los coronavirus los cuales se identifican mediante las letras S (homotrímero de glicoproteína que forman las puntas de la superficie), E (proteína de bajo tamaño de la envoltura), M (proteína de la matriz que une la envoltura con el núcleo) y N (fosfoproteína de la nucleocápside), además de los marcos de lectura abiertos que codifican proteínas no estructurales en las que encontramos, las enzimas causantes de su ciclo reproductivo intrahospedero. \\
		
		Toda esta información sobre el virus era desconocida en diciembre de 2019 cuando apareció en Wuhan , provincia de Hubei (China). Un brote epidémico de lo que se llamaba neumonía por causa desconocida que llegó a afectar a más de 60 personas durante ese mes. Esto es debido a que el coronavirus puede infectar de manera selectiva las mucosas pulmonares o grastrointestinales. La forma de acceder a una célula epitelial es mediante un receptor presente en las superficie del organismo que recibe el nombre de ACE2. Dichos receptores son más comunes ser encontrados en los pulmones, por ello esta enfermedad está considerada de tipo respiratorio. El sistema inmunológico humano contraataca con una respuesta dura, liberando interferones cuya función es dificultar la replicación del virus dentro de las células epiteliales.\\
		
		
		En este trabajo estudiaremos la respuesta desarrollada en las células del epitelio del pulmón a la infección por SARS-Cov2 mediante el análisis de los perfiles de expresión génica publicados en el dataset GEO GSE147507. Estos perfiles de expresión se analizarán mediante el modelado de redes de coexpresión génica, con el paquete de R WGCNA.
		
\end{document}

	


	
	\section{Materiales y métodos}
	
	Para llevar a cabo el modelado de redes de coexpresi´n génica se ha utilizado el lenguaje R mediante el entorno RStudio. Usaremos el paquete WGCNA, también conocido como análisis de coexpresion de genes ponderados. Este es un método de biología de sistemas utilizado para decribir los patrones de correlación entre genes en muestras de microarrays. Las redes de correlacion facilitan los métodos de cribado e genes basados en redes que se pueden utilizar para identificar posibles biomarcadores o dianas terapeuticas
	
	El paquete utilizado es una colección completa de funciones R que incluye funciones para la construcción de redes, detección de módulos, selección de genes, cálculos de propiedades topológicas, simulación de datos, visualización e interfaz con software externo. En este caso, utilizaremos la funcion FUNCION1 que se encargará de ....
	
	Para la obtención del paquete WCGNA utilizaremos el paquete BiocManager como paquete que permite a los usuarios instalar y administrar paquetes del proyecto BioConductor. El uso de los paquetes BiocManager, permite a los usuarios instalar con precisión los paquetes de la versión adecuada.
	
	-paquete RTCGAToolbox
	
	-paquete DESeq2
	
	-paquete ggplot2
	
	-paquete dplyr
	
	-paquete cluster
	
	-paquete nbclust
	
	-paquete factoextra
	
	-paquete DCGL
	



	
\graphicspath{{/Users/laura/Desktop/BioSis/project_template_BS_SARS-CoV2/results/}}

\newpage

\section{Resultados}



\noindent El primer resultado obtenido corresponde con un análisis de la topología de la red para varias potencias de umbral suave. El panel de la izquierda muestra el ajuste de escala libre, correspondiente al índice (eje y), en función de la potencia de umbral suave, que sería el (eje x). El panel derecho muestra la conectividad media (grados, eje y) en funcion de la potencia de umbral suave (eje x).


	\begin{figure}[!htb]
		\includegraphics[width=0.9\textwidth]{escalaIndep.jpg}
		\caption{Escala de Independencia}
		\label{fig:cost_megabase}
	\end{figure}


\noindent En la siguiente figura podemos ver un dendograma que expresa un clustering TOM disimilitud, y abajo las dos formas de obtener los módulos, primero mediante un árbol de corte dinámico y el otro mediante fusión. Cada color corresponde a un modulo.

	\begin{figure}[!htb]
		\includegraphics[width=0.9\textwidth]{mergeDinamic.jpg}
		\caption{Dendograma de función Merge}
		\label{fig:cost_megabase}
	\end{figure}

\noindent En la figura 3 tenemos un mapa de calor que nos expresa la correlación de los módulos expresados mediante corte de  árbol dinámico.

	\begin{figure}[!htb]
		\includegraphics[width=0.9\textwidth]{mapaCalor.jpg}
		\caption{Mapa Calor}
		\label{fig:cost_megabase}
	\end{figure}


\noindent En la figura 4 tenemos otro dendograma en el cual están agrupados por los módulos obtenidos anteriormente. A continuación, se encuentra su matriz de adyacencia.

\begin{figure}[!htb]
	\includegraphics[width=0.9\textwidth]{eigengene.jpg}
	\caption{Gráfica de adyacencia}
	\label{fig:cost_megabase}
	\end{figure}

\newpage

\noindent Por último, en la figura 5, mostramos la red de correlación de las muestras y sus genes, en respuesta de las células epiteliales del pulmón a la infección por SARS-CoV2. Nos indican la relación que existe entre las muestras.

	\begin{figure}[!htb]
		\includegraphics[width=0.9\textwidth]{cor_pearson.jpeg}
		\caption{Red de Correlación}
		\label{fig:cost_megabase}
	\end{figure}


	
\newpage

\section{Discusión}

\noindent Para la correcta realización de nuestro estudio, al tener un número de datos tan grande, se ha requerido realizar una serie de filtrados. Tras una profunda revisión de los datos, nos dimos cuenta que habıa muchos valores nulos, que en este caso se corresponden con valores 0, por lo que primero comenzamos eliminando aquellas columnas que tengan más de un 51\% de 0. El numero de variables se ha reducido en torno a un 70\%
 \\

\noindent Una vez realizado el filtrado, dado que todavía tenemos una enorme cantidad de valores 0, sustituiremos los 0 por la media de las columnas. A continuación, ya podremos realizar nuestro estudio.
 \\

\noindent Comenzaremos realizando un Sample Clustering, el cuál es un método de muestreo donde se crea mútiples grupos de genes de una población donde son indicativos de características homógeneas y tienen la misma probabilidad de ser parte de la muestra. Se encarga de agrupar. Después realizamos la escapa de independencia, el valor 3 indica que se encuentra en torno al 50\% indicaría que tiene ese porcentaje de probabilidades de seguir un modelo de escala libre topológica. Como podemos observar, sigue una relación inversa con la conectividad media, ya que en la conectividad media, conforme aumenta el umbral suave disminuye la media de la conectividad.
\\

\noindent Hemos usado WGCNA para obtener los datos de expresión que han sido obtenidos mediante la separación de los clusters y a cada cluster, le aplicamos el análisis de coexpresión y vemos cuales tiene mayor correlación con la coexpresión del cluster 1.
 \\

\noindent Después, pasamos a hacer un clustering jerérquico basado de la matriz de medición de superposición topológica (TOM) para los datos de expresión génica. La gráfica de medida de superposición topológica muestra grupos de genes (módulos) altamente interconectados. Los genes se asignaron a módulos nombrados por los colores debajo del dendrograma utilizando el método de corte de árbol dinámico.
 \\

\noindent Una vez que hemos obtenido los módulos, realizaremos un mapa de calor que nos mostrará la correlación de cada uno de los módulos. Cuanto mayor es la correlación, más claro será el color. También obtenemos un mapa de adyacencia, cuanto más se acerca al color rojo más adyacentes son los módulos.
 \\

\noindent Por último, realizamos una red de correlación que nos indica la relación de los genes.
	\newpage
\section{Conclusiones}

\noindent Basándonos en los resultados obtenidos y en las diferentes fuentes,  podemos llegar a la conclusión de que los módulos más largos y comunes, tanto la obtención por corte de árbol dinámico como por fusión, son aquellos que se sobreexpresan o activan a la infección por SARS-CoV2 en las células epiteliales del pulmón. Siendo uno de los genes más activados el IL-6 indicando una infección, trauma o hemorragia en nuestras muestras. Además, también podemos concluir que ciertas citoquinas se muestran elevadas y los niveles de interferones tipo I y tipo III (IFN-I, III) tienen un bajo nivel de activación. \\

\noindent Para finalizar, nos hemos dado cuenta que el epitelio, además de tener un papel estructural, es un claro indicador de la respuesta inmune.
	
	
	%%%%%%%%%%%%%%%%%%%%%%%%%%%%%%%%%%%%%%%%%%%%%%
	%% OTRA INFORMACIÓN                         %%
	%%%%%%%%%%%%%%%%%%%%%%%%%%%%%%%%%%%%%%%%%%%%%%
	
	\begin{backmatter}
	
		\section*{Abreviaciones}%% if any
			Indicar lista de abreviaciones mostrando cada acrónimo a que corresponde
		
		\section*{Disponibilidad de datos y materiales}%% if any
			Debéis indicar aquí un enlace a vuestro repositorio de github.
		
		\section*{Contribución de los autores}
			Usando las iniciales que habéis definido al comienzo del documento, debeis indicar la contribución al proyecto en el estilo:
			J.E : Encargado del análisis de coexpresión con R, escritura de resultados; J.R.S : modelado de red con python y automatizado del código, escritura de métodos; ...
			OJO: que sea realista con los registros que hay en vuestros repositorios de github. 
		
		
		%%%%%%%%%%%%%%%%%%%%%%%%%%%%%%%%%%%%%%%%%%%%%%%%%%%%%%%%%%%%%%%%%%%%%%%%%%%%%%%%%%%%%%%%
		%% BIBLIOGRAFIA: no teneis que tocar nada, solo sustituir el archivo bibliography.bib %%
		%% por el que hayais generado vosotros                                                %%
		%%%%%%%%%%%%%%%%%%%%%%%%%%%%%%%%%%%%%%%%%%%%%%%%%%%%%%%%%%%%%%%%%%%%%%%%%%%%%%%%%%%%%%%%
		
		\bibliographystyle{bmc-mathphys} % Style BST file (bmc-mathphys, vancouver, spbasic).
		\bibliography{bibliography}      % Bibliography file (usually '*.bib' )
	
	\end{backmatter}
\end{document}
