\newpage
\section{Conclusiones}

\noindent Basándonos en los resultados obtenidos y en las diferentes fuentes,  podemos llegar a la conclusión de que los módulos más largos y comunes, tanto la obtención por corte de árbol dinámico como por fusión, son aquellos que se sobreexpresan o activan a la infección por SARS-CoV2 en las células epiteliales del pulmón. Siendo uno de los genes más activados el IL-6 indicando una infección, trauma o hemorragia en nuestras muestras. Además, también podemos concluir que ciertas citoquinas se muestran elevadas y los niveles de interferones tipo I y tipo III (IFN-I, III) tienen un bajo nivel de activación. \\

\noindent Para finalizar, nos hemos dado cuenta que el epitelio, además de tener un papel estructural, es un claro indicador de la respuesta inmune.