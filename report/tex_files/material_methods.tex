
	\section{Materiales y métodos}
	
	Para llevar a cabo el modelado de redes de coexpresi´n génica se ha utilizado el lenguaje R mediante el entorno RStudio. Usaremos el paquete WGCNA, también conocido como análisis de coexpresion de genes ponderados. Este es un método de biología de sistemas utilizado para decribir los patrones de correlación entre genes en muestras de microarrays. Las redes de correlacion facilitan los métodos de cribado e genes basados en redes que se pueden utilizar para identificar posibles biomarcadores o dianas terapeuticas
	
	El paquete utilizado es una colección completa de funciones R que incluye funciones para la construcción de redes, detección de módulos, selección de genes, cálculos de propiedades topológicas, simulación de datos, visualización e interfaz con software externo. En este caso, utilizaremos la funcion FUNCION1 que se encargará de ....
	
	Para la obtención del paquete WCGNA utilizaremos el paquete BiocManager como paquete que permite a los usuarios instalar y administrar paquetes del proyecto BioConductor. El uso de los paquetes BiocManager, permite a los usuarios instalar con precisión los paquetes de la versión adecuada.
	
	-paquete RTCGAToolbox
	
	-paquete DESeq2
	
	-paquete ggplot2
	
	-paquete dplyr
	
	-paquete cluster
	
	-paquete nbclust
	
	-paquete factoextra
	
	-paquete DCGL
	


